%% This is file `elsarticle-template-5-harv.tex',
%%
%% Copyright 2009 Elsevier Ltd
%%
%% This file is part of the 'Elsarticle Bundle'.
%% ---------------------------------------------
%%
%% It may be distributed under the conditions of the LaTeX Project Public
%% License, either version 1.2 of this license or (at your option) any
%% later version.  The latest version of this license is in
%%    http://www.latex-project.org/lppl.txt
%% and version 1.2 or later is part of all distributions of LaTeX
%% version 1999/12/01 or later.
%%
%% The list of all files belonging to the 'Elsarticle Bundle' is
%% given in the file `manifest.txt'.
%%
%% Template article for Elsevier's document class `elsarticle'
%% with harvard style bibliographic references
%%
%% $Id: elsarticle-template-5-harv.tex 159 2009-10-08 06:08:33Z rishi $
%% $URL: http://lenova.river-valley.com/svn/elsbst/trunk/elsarticle-template-5-harv.tex $
%%
\documentclass[preprint,authoryear,12pt]{elsarticle}

%% Use the option review to obtain double line spacing
%% \documentclass[authoryear,preprint,review,12pt]{elsarticle}

%% Use the options 1p,twocolumn; 3p; 3p,twocolumn; 5p; or 5p,twocolumn
%% for a journal layout:
%% \documentclass[final,authoryear,1p,times]{elsarticle}
%% \documentclass[final,authoryear,1p,times,twocolumn]{elsarticle}
%% \documentclass[final,authoryear,3p,times]{elsarticle}
%% \documentclass[final,authoryear,3p,times,twocolumn]{elsarticle}
%% \documentclass[final,authoryear,5p,times]{elsarticle}
%% \documentclass[final,authoryear,5p,times,twocolumn]{elsarticle}

%% if you use PostScript figures in your article
%% use the graphics package for simple commands
%% \usepackage{graphics}
%% or use the graphicx package for more complicated commands
%% \usepackage{graphicx}
%% or use the epsfig package if you prefer to use the old commands
%% \usepackage{epsfig}

%% The amssymb package provides various useful mathematical symbols
\usepackage{amssymb}
%% The amsthm package provides extended theorem environments
%% \usepackage{amsthm}

\usepackage{siunitx}

%% The lineno packages adds line numbers. Start line numbering with
%% \begin{linenumbers}, end it with \end{linenumbers}. Or switch it on
%% for the whole article with \linenumbers after \end{frontmatter}.
%% \usepackage{lineno}

%% natbib.sty is loaded by default. However, natbib options can be
%% provided with \biboptions{...} command. Following options are
%% valid:

%%   round  -  round parentheses are used (default)
%%   square -  square brackets are used   [option]
%%   curly  -  curly braces are used      {option}
%%   angle  -  angle brackets are used    <option>
%%   semicolon  -  multiple citations separated by semi-colon (default)
%%   colon  - same as semicolon, an earlier confusion
%%   comma  -  separated by comma
%%   authoryear - selects author-year citations (default)
%%   numbers-  selects numerical citations
%%   super  -  numerical citations as superscripts
%%   sort   -  sorts multiple citations according to order in ref. list
%%   sort&compress   -  like sort, but also compresses numerical citations
%%   compress - compresses without sorting
%%   longnamesfirst  -  makes first citation full author list
%%
%% \biboptions{longnamesfirst,comma}

% \biboptions{}

\journal{Biosystems Engineering}

\begin{document}

\begin{frontmatter}

%% Title, authors and addresses

%% use the tnoteref command within \title for footnotes;
%% use the tnotetext command for the associated footnote;
%% use the fnref command within \author or \address for footnotes;
%% use the fntext command for the associated footnote;
%% use the corref command within \author for corresponding author footnotes;
%% use the cortext command for the associated footnote;
%% use the ead command for the email address,
%% and the form \ead[url] for the home page:
%%
%% \title{Title\tnoteref{label1}}
%% \tnotetext[label1]{}
%% \author{Name\corref{cor1}\fnref{label2}}
%% \ead{email address}
%% \ead[url]{home page}
%% \fntext[label2]{}
%% \cortext[cor1]{}
%% \address{Address\fnref{label3}}
%% \fntext[label3]{}

\title{An Autonomous Platform for Use in Kiwifruit Orchards}

%% use optional labels to link authors explicitly to addresses:
%% \author[label1,label2]{<author name>}
%% \address[label1]{<address>}
%% \address[label2]{<address>}

% \author{Mark H. Jones, Jamie Bell, Matthew Seabright, Joshua Barnett, Alistair Scarfe, Bruce MacDonald, Mike Duke}

%% Group authors per affiliation:
\author[UoW]{Mark H. Jones\corref{mjemail}} 
\cortext[mjemail]{markj@waikato.ac.nz}

\author[UoA]{Jamie Bell\corref{jbemail}}
\cortext[jbemail]{jamie977@gmail.com}
\author[UoW]{Matthew Seabright}
\author[UoW]{Josh Barnett}
\author[RPL]{Alistair Scarfe}
\author[UoW]{Mike Duke}
\author[UoA]{Bruce MacDonald}

\address[UoW]{School of Engineering, University of Waikato, Hamilton, New Zealand}
\address[UoA]{Faculty of Engineering, University of Auckland, Auckland, New Zealand}
\address[RPL]{Robotics Plus Ltd, Newnham Innovation Park, Tauranga, New Zealand}

\begin{abstract}
%% Text of abstract

    This will be written last.
    Other uses may be mechanical weed removal, targeted spraying, and crop scouting.
\end{abstract}

\begin{keyword}
%% keywords here, in the form: keyword \sep keyword

%% MSC codes here, in the form: \MSC code \sep code
%% or \MSC[2008] code \sep code (2000 is the default)

\end{keyword}

\end{frontmatter}

% \linenumbers

%% main text
\section{Introduction}
\label{sect:intro}
    Short-term labor requirements within the New Zealand kiwifruit industry peak twice a year corresponding with pollination and harvesting cycles.
    The majority of employment in this industry during these peaks is filled by seasonal or casual workers \citep{Timmins2009}.
    As kiwifruit is New Zealand's largest horticultural export by value \citep{StatisticsNewZealand2015}, automation in kiwifruit harvesting and pollination should ease growth in this industry.
    Additionally, the New Zealand government aims to double exports from its primary industries between 2012 and 2025 and is actively investing in programmes to achieve this \citep{MinistryPrimaryIndustries2015}.

    Previous work on automated harvesting of kiwifruit has been demonstrated \citep{Scarfe2012,scarfe2009}.
    That work presents a harvesting platform with the capability of in-orchard kiwifruit harvesting from pergola type orchards.
    The robotic platform presented in this paper is a second generation unit based on that previous, more integrated, design of kiwifruit harvester.
    Modularity of the platform has been increased to so as to be able to accommodate other modules, such as those for harvesting and pollination.
    This work discusses only the base platform, where details of the harvesting and pollination modules are published separately \citep{williams2017,Seabright2017}.

    Automation in harvesting and pollinating kiwifruit demands the use of real-time computer control, state-of-the-art manipulators, and convolutional neural networks.
    In their current state of development these systems are bulky and have specific geometric requirements dictated by the environment they operate in, namely pergola style orchards.
    They share common requirements in that they both require transport to and from orchards, electrical power, and air pressure, but differ in they way they move the orchard.
    Whilst pollinating, the platform must move at a well-known velocity with minimum changes in angle, whereas harvesting repeatedly starts and stops, advancing set distance between cycles.
    The duration of any given harvesting cycle is determined by the number of fruit to be harvested during that particular cycle.
    Therefore, as the harvester is designed to be autonomous, there must be communication between the harvester and platform to trigger forward movement after each harvest cycle.

    Many autonomous vehicles for use in the agricultural industry have previously been reported, many of those being conversions of existing vehicles into a self-driving form.
    While this can reduce the initial cost for development of self driving systems, the cost of deploying such units may make them commercially unfeasible.
    In this work we present an agricultural vehicle designed specifically for use in kiwifruit orchards to transport modularised pollinating and harvesting units.
    The weight and size of these units requires a vehicle that is robust, computer controllable, low-slung, and is capable of performing turns between adjacent orchard rows.

    It has been stated that ``since the robot development already includes a high complexity, the application itself should be of comparably low complexity'' \citep{Ruckelshausen2009}.
    By separating development the base platform from the more specialised task-specific modules, risk of over-complexity is somewhat reduced.
    The platform presented here has one simple application in that it transports the separately developed modules through kiwifruit orchards.
    Thus, in order for the unit to be successful in its own right, it only needs to demonstrate the ability to navigate through kiwifruit orchards autonomously.

\section{Review}
\label{sect:review}

    The introduction of computers and digital camera technology during the 1980s sparked research into creating autonomous vehicles for agricultural use \cite{Li2009}.
    When publishing details of an autonomous vehicle in 1999, Tillett et al. cites difficulties dealing with variability in lighting and the environment as the reason no commercial ready vehicles were available at the time.
    His vehicle used wheel encoders, a compass, and accelerometers for odometry information, and featured a camera based row guidance system.
    It was capable of spraying individual plants whilst autonomously driving at \SI{2.5}{\kilo\meter\per\hour}.
    
    Later, in 2002, \cite{Pedersen2002} presented an Ackermann based autonomous robot designed for weed mapping.
    It had a top speed of \SI{6}{\kilo\meter\per\hour} and used GPS to determine the absolute position of the vehicle at semi-regular intervals.
    Between those updates, odometry estimates based on a gyro and compass were used.
    Their vehicle was designed to follow pre-defined paths through row-crops, but the authors found that this was impractical without a separate row sensor.

    These reports suggest that neither perception based row guidance nor absolute positioning, such as by GPS, are suitable on their own for agricultural vehicle guidance.
    
    \cite{Bak2004} presented a relatively advanced robotic platform based on a four wheel steering geometry.
    It is noted that the control strategy for the four independently controlled wheels was non-trivial.
    Like the platform presented earlier by Pederson et~al.\@, it combined a compass, gyroscope and GPS for odommetry.
    However, it also featured encoder feedback, a row detection sensor and a GPS unit utilising Real Time Kinematic (RTK) corrections from a base station.
    RTK-GPS is capable of providing positioning with accuracies of around \SI{2}{\centi\meter}.
    Their robot utilised a Controller Area Network (CAN) bus for some aspects of system communication.

    In 2008, Klose et~al.\@ publish details of `Weedy', a autonomous weed control robot for field use.
    It used a simplistic four wheel steering geometry.
    There are few details on the sensor selection apart from mention of the use of cameras and 'acoustic distance sensors'.
    Presumably the selection of drive geometry on this robot is a cost/complexity optimisation.
    It too makes use of a CAN bus for communication between on-board modules.

    The following year, many the same authors appearing on the `Weedy' paper published details an autonomous robotic platform with four wheel steering named BoniRob \citep{Ruckelshausen2009}.
    BoniRob had the ability raise and lower itself and alter its wheel placement by actuating the arms to which the motors are attached to.
    Similar to the unit presented by Bak et~al.\@ it features a gyroscope and RTK-GPS for localisation.
    It introduces the use of both 2D and 3D laser-scanning (or lidar) for perception and row detection.
    A CAN bus is used to control the low level systems (such as the drive control) and ethernet connections for higher level communication.
    The authors created a simulated model of the platform using Gazebo in which they could test the many-degrees-of-freedom drive system.

    Of particular relevance is the work of Scarfe et~al.\@ on an autonomous kiwifruit picking robot \citep{scarfe2009, Scarfe2012}.
    That work involved the creation of a hydraulically driven platform with Ackermann steering to which four fruit harvesting arms were integrated.
    For navigation it used lidar, camera based machine vision, GPS, and a compass.
    The development of that robot forms much of the foundation for the work presented here.

    Common among these vehicles is the use of sensor fusion, whereby data from multiple sensors is merged and filtered.
    This provides a way to combine the advantages of multiple sensor types, and the benefit of redundancy from multiple sensors, into a single computation space.
    One such sensor is the RTK-GPS unit, appearing on platforms published after 2004.
    With regards to its use in place of machine vision based guidance systems, Slaughter et~al.\@ points out the trade-off of requiring an ``unobstructed ``view'' of the sky from all parts of the field'' \citep{Slaughter2008}.
    This requirement can not be satisfied under the dense canopy of a kiwifruit orchard which are generally surrounded by thick, and tall wind-breaking hedges.
    Additionally, multi-path signal propagation caused by nearby foliage or the geometry of the land itself presents its own mode of failure \citep{Durrant-Whyte2005}.
    A separate feasibility analysis highlighted the use of RTK-GPS systems as a significant cost in yearly subscriptions alone \citep{Pedersen2006}.
    Torii suggests a combination of both RTK-GPS and machine vision systems to be the most promising system going forward based on reductions in costs and increases in performance of these systems \cite{Torii2000}.

    Of the reported platforms there appears to be an equal split between Ackermann and four-wheel steering geometries.
    The ability of four wheel steering to pivot about the centre of the vehicle, as opposed to the centre of the rear wheels with an Ackermann system offers little gain in an orchard environment.
    Not only does Ackermann offer a reduction in mechanical complexity, but removes the need to develop the ``non-trivial'' control strategies.

    \cite{Blackmore2007} envisaged significant reductions in production costs by re-purposing parts already in use in the agricultural and automotive industry.
    While not a physical component, the CAN is one such technology borrowed from the automotive industry aiding developmental of low-level communications.
    Many of the platforms reviewed, especially the more recent ones, made use of this protocol for real-time communication.

    The use of simulation tools allowed the creators of BoniRob to develop and test their mobility system separate of the physical hardware.
    The Gazebo robot simulation software used by that group is open source and has become tightly integrated into the Robot Operating System (ROS).

    Trends in the development of similar platforms show a increase in the variety of sensors used.
    With that, comes advances in the way the sensor data is combined into a single frame of reference.
    CAN as a means of low-level communication gained popularity and appears to be the standard signaling protocol.
    Platforms designed for open field crops appear to favor four-wheel steering over the more traditional Ackermann geometry.

\section{Mechanical design}
\label{sect:mechanical}

    As the platform is intended solely for use in kiwifruit orchards, its design is specific to this environment.
    An orchard canopy can 
    The platform is designed solely for use in kiwifruit orchards, which dictates much 
    where canopy heights can be as low as \SI{1300}{\milli\meter}

    \begin{figure}[htbp]
        \centering
        \includegraphics[width=\linewidth]{imgs/profile_views/AMMP-All-Labelled.pdf}
        \caption{Profile drawings of the robotic platform with kiwifruit bin.}
        \label{fig:AMMP}
    \end{figure}

    Skid steering inappropriate for orchard environments due to soft ground.
    Pivoting front axle used to maintain three points of contact at all times; no suspension.
    Limited to \SI{10}{\kilo\meter\per\hour} by the choice of motor and gearboxes on the drive-system.
    Operational speed of \SI{5}{\kilo\meter\per\hour}.
    Ackermann steering geometry with the ability to pivot about the centre of the rear wheels.
    Four wheel steering, such as presented by Bak \& Hans \citep{Bak2004}, was not deemed necessary as the headlands of kiwifruit orchards provide adequate turning areas.
    Additionally, control strategies for a platform having only two steerable wheels is simpler.
    % Diagram showing turn circle of four wheel steer vs two wheel steer, showing only a half wheelbase difference.



    \citep{Astrand2002} have used an ackermann steering system actuated by a single DC servo motor for their robotic beet-crop weeding platform. They have only two driving wheels placed at the rear of the system.


\section{Hardware}
\label{sect:hardware}

    \begin{figure}[htbp]
        \centering
        \includegraphics[width=\linewidth]{imgs/photos/development_wBatteryBoxes.jpg}
        \caption{
            Photo of the platform during development showing battery housings and system internals.
            Battery compartments are visible along the sides of the chassis.
        }
        \label{fig:AMMP}
    \end{figure}

    GPS has proven to be unreliable when used under the dense canopy of a kiwifruit orchard.
    \citep{Pedersen2006} shows the economics of using an GPS-RTK system, as seen in other agricultural systems \citep{Bak2004,Ruckelshausen2009}[Nagasaka et al from \citep{Torii2000}], has considerable ongoing costs in the form of yearly fees, although the cost of these units is rapidly decreasing \citep{Torii2000}.
    Forward and upwards facing LiDAR have been used for navigation and detection of the row and canopy.

\section{Safety}
\label{sect:safety}
    % Include system layout diagram
    Relay modules connected to the main computer via the system's CAN bus give a means of shutting down subsystems.
    These relays monitor the platform's CAN bus to ensure that synchronisation messages are being sent out in a timely manner.
    In the event that the synchonisation messages begin to vary in frequency, or stop, the relays cut power to the subsystems.
    Both front drive motors are fitted with electromechanical brakes which engage when the power is cut.
    A wireless safety-rated controller, designed for use with cranes, has been adapted for use with driving robot platform.
    The controller provides the operator with a way of entering the platform into autonomous mode, manual control, triggering an emergency stop, or enabling/disabling auxiliary systems.

\section{Software architecture}
\label{sect:software}
    The control software is comprised of individual nodes, writen in either C++ or Python, linked together using Robot Operating System (ROS) for interprocess communication.
    The system runs on Ubuntu Server 16.04 on an Intel NUC, a compact x86 based PC.
    A model of the robot platform has been depeloped for use with Gazebo simulation software.
    Such a model provides a way to test steering and movement strategies before deploying them on the hardware.

\section{Sensor selection}
\label{sect:sensors}
    [Jamie's section]

\section{Random info}

``Field scouting and mechanical weeding have been identified and described as the first two niche tasks likely to become autonomous''\citep{Blackmore2004}.


% Diagram of sensor placement
% Lidar units and visibility
% All electric drive system based on ackermann steering
% Sevcons modified to allow for direct CAN control.
% Power system and battery pack
% power-gen capable of charging battery pack approx 24hour duty
% Drive motors used as steering motors
% Control based on ROS, complete with gazebo simulation
% State the turning circle.
% State the mass and payload capability
% Show the area for drop-on modules (diagram)
% Integrated bin lifting mechanism
% Platform electronics

% Sections:
%   Mechanical design
%   Hardware and sensors
%   Software architecture


%% The Appendices part is started with the command \appendix;
%% appendix sections are then done as normal sections
%% \appendix

%% \section{}
%% \label{}

%% References
%%
%% Following citation commands can be used in the body text:
%%
%%  \citet{key}  ==>>  Jones et al. (1990)
%%  \citep{key}  ==>>  (Jones et al., 1990)
%%
%% Multiple citations as normal:
%% \citep{key1,key2}         ==>> (Jones et al., 1990; Smith, 1989)
%%                            or  (Jones et al., 1990, 1991)
%%                            or  (Jones et al., 1990a,b)
%% \cite{key} is the equivalent of \citet{key} in author-year mode
%%
%% Full author lists may be forced with \citet* or \citep*, e.g.
%%   \citep*{key}            ==>> (Jones, Baker, and Williams, 1990)
%%
%% Optional notes as:
%%   \citep[chap. 2]{key}    ==>> (Jones et al., 1990, chap. 2)
%%   \citep[e.g.,][]{key}    ==>> (e.g., Jones et al., 1990)
%%   \citep[see][pg. 34]{key}==>> (see Jones et al., 1990, pg. 34)
%%  (Note: in standard LaTeX, only one note is allowed, after the ref.
%%   Here, one note is like the standard, two make pre- and post-notes.)
%%
%%   \citealt{key}          ==>> Jones et al. 1990
%%   \citealt*{key}         ==>> Jones, Baker, and Williams 1990
%%   \citealp{key}          ==>> Jones et al., 1990
%%   \citealp*{key}         ==>> Jones, Baker, and Williams, 1990
%%
%% Additional citation possibilities
%%   \citeauthor{key}       ==>> Jones et al.
%%   \citeauthor*{key}      ==>> Jones, Baker, and Williams
%%   \citeyear{key}         ==>> 1990
%%   \citeyearpar{key}      ==>> (1990)
%%   \citetext{priv. comm.} ==>> (priv. comm.)
%%   \citenum{key}          ==>> 11 [non-superscripted]
%% Note: full author lists depends on whether the bib style supports them;
%%       if not, the abbreviated list is printed even when full requested.
%%
%% For names like della Robbia at the start of a sentence, use
%%   \Citet{dRob98}         ==>> Della Robbia (1998)
%%   \Citep{dRob98}         ==>> (Della Robbia, 1998)
%%   \Citeauthor{dRob98}    ==>> Della Robbia


%% References with bibTeX database:

\bibliographystyle{model5-names}
\bibliography{library}

%% Authors are advised to submit their bibtex database files. They are
%% requested to list a bibtex style file in the manuscript if they do
%% not want to use model5-names.bst.

%% References without bibTeX database:

% \begin{thebibliography}{00}

%% \bibitem must have one of the following forms:
%%   \bibitem[Jones et al.(1990)]{key}...
%%   \bibitem[Jones et al.(1990)Jones, Baker, and Williams]{key}...
%%   \bibitem[Jones et al., 1990]{key}...
%%   \bibitem[\protect\citeauthoryear{Jones, Baker, and Williams}{Jones
%%       et al.}{1990}]{key}...
%%   \bibitem[\protect\citeauthoryear{Jones et al.}{1990}]{key}...
%%   \bibitem[\protect\astroncite{Jones et al.}{1990}]{key}...
%%   \bibitem[\protect\citename{Jones et al., }1990]{key}...
%%   \harvarditem[Jones et al.]{Jones, Baker, and Williams}{1990}{key}...
%%

% \bibitem[ ()]{}

% \end{thebibliography}

\end{document}

%%
%% End of file `elsarticle-template-5-harv.tex'.
